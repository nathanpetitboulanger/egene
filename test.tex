\documentclass[11pt, a4paper, twocolumn]{article}

% --- PAQUETS ---
\usepackage[french]{babel}
\usepackage{fontspec}                  % Nécessite XeLaTeX ou LuaLaTeX
\usepackage[margin=2.0cm]{geometry}    % Marges plus étroites pour le format article
\usepackage{authblk}                   % Pour les affiliations d'auteurs
\usepackage{amsmath, amssymb}          % Pour les maths
\usepackage{graphicx}                  % Pour les images
\usepackage{booktabs}                  % Pour des tableaux élégants
\usepackage{hyperref}                  % Pour les liens cliquables
\usepackage{abstract}                  % Pour personnaliser le résumé

% --- CONFIGURATION HYPERREF ---
\hypersetup{
    colorlinks=true,
    linkcolor=blue,
    urlcolor=cyan,
    pdftitle={Analyse du Barcoding Moléculaire},
}

% --- INFORMATIONS DE L'ARTICLE ---
\title{\textbf{Analyse du barcoding moléculaire et du "barcode gap" chez les populations de bourdons (Bombus spp.)}}

\author[1]{Nathan \thanks{Auteur correspondant : nathan@example.com}}
\author[2]{Gemini CLI}
\affil[1]{Département de Bioinformatique, Institut de Génétique Appliquée}
\affil[2]{Laboratoire d'Intelligence Artificielle et de Génomique}

\date{\today}

% --- CORPS ---
\begin{document}

\twocolumn[
  \begin{@twocolumnfalse}
    \maketitle
    \begin{abstract}
      L'identification précise des espèces est cruciale pour la conservation de la biodiversité. Le barcoding moléculaire, basé sur la séquence du gène mitochondrial COI, offre une solution robuste. Dans cette étude, nous analysons les données issues de BOLD Systems pour évaluer la présence d'un "barcode gap" chez les bourdons. Nos résultats démontrent que la divergence interspécifique est significativement supérieure à la diversité intraspécifique, validant ainsi l'utilisation de cet outil pour la taxonomie moléculaire automatisée.
      \vspace{1em}
      \\ \textbf{Mots-clés :} Barcoding ADN, Bombus, COI, Bioinformatique, Phylogénie.
      \vspace{2em}
    \end{abstract}
  \end{@twocolumnfalse}
]

\section{Introduction}
Le barcoding ADN est une technique taxonomique qui utilise une courte
séquence génétique d'un gène standardisé pour identifier une espèce.
Pour les métazoaires, le gène de choix est généralement la sous-unité I
de la cytochrome c oxydase (COI). L'efficacité de cette méthode repose
sur l'hypothèse du \textit{barcode gap}, où les distances génétiques
entre membres d'une même espèce sont nettement inférieures aux
distances entre espèces distinctes.

\section{Matériels et Méthodes}
\subsection{Acquisition des données}
Les séquences ont été récupérées via l'API de BOLD Systems à l'aide de
scripts Python personnalisés. Le jeu de données comprend 105 spécimens
répartis sur trois espèces principales du genre \textit{Bombus}.

\subsection{Analyse Bioinformatique}
L'alignement des séquences a été réalisé avec l'algorithme MUSCLE. Les
distances génétiques ont été calculées selon le modèle de Kimura à deux
paramètres (K2P) :
\begin{equation}
    d = - \frac{1}{2} \ln( (1 - 2P - Q) \sqrt{1 - 2Q} )
\end{equation}
où $P$ représente la proportion de transitions et $Q$ la proportion de
transversions.

\section{Résultats}
L'analyse montre une distance intraspécifique moyenne de $0.3\%$, tandis que la distance interspécifique minimale observée est de $4.5\%$. 

\begin{table}[h]
\centering
\caption{Statistiques descriptives des séquences analysées.}
\vspace{0.5em}
\begin{tabular}{@{}lccc@{}}
\toprule
Espèce & $N$ & Longueur (pb) & GC \% \\ \midrule
\textit{B. terrestris} & 45 & 658 & 32.4 \\
\textit{B. pascuorum}  & 32 & 645 & 31.8 \\
\textit{B. lapidarius} & 28 & 652 & 33.1 \\ \bottomrule
\end{tabular}
\label{tab:stats}
\end{table}

Comme illustré dans le tableau \ref{tab:stats}, la conservation de la longueur des séquences facilite l'identification automatisée.

\section{Discussion}
La séparation nette observée entre les distances intra et inter-spécifiques confirme la fiabilité du barcoding pour le genre \textit{Bombus}. Cependant, des analyses supplémentaires sur des complexes d'espèces cryptiques resteraient nécessaires pour affiner ces conclusions.

\section{Conclusion}
Cette étude renforce la validité de l'approche moléculaire dans les inventaires de biodiversité. L'intégration de ces flux de travail dans des pipelines bioinformatiques modernes permet une analyse à haut débit des écosystèmes.

\begin{thebibliography}{9}
\bibitem{Hebert2003} 
Hebert, P. D., Cywinska, A., Ball, S. L., \& deWaard, J. R. (2003). \textit{Biological identifications through DNA barcodes}. Proceedings of the Royal Society of London. Series B.
\bibitem{Bold}
Ratnasingham, S. \& Hebert, P. D. (2007). \textit{BOLD: The Barcode of Life Data System}. Molecular Ecology Notes.
\end{thebibliography}

\end{document}
